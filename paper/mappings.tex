There are many language features in modern object-oriented programming
languages, which do not exist in \eo{}, for example:
  multiple inheritance,
  annotations,
  encapsulation and information hiding,
  mixins and traits,
  constructors,
  classes,
  assertions,
  static blocks,
  aspects,
  generics,
  lambda functions,
  exception handling,
  reflection and type casting,
  and so on.
We assume that all of them may be represented with the primitive
language features of \eo{}. There is no complete mapping mechanism
implemented yet, but there are a few examples in this section
that demonstrate how some features may be mapped from Java to \eo{}.
The Tab.~\ref{tab:mapping} shows the list of most popular Java~8 features and explains
how they may be mapped to \eo{}.

\begin{figure}
\begin{tabularx}{\columnwidth}{|>{\raggedright\arraybackslash}p{7em}|>{\raggedright\arraybackslash}X|}
\hline
Java~8 Feature & Equivalent implementation in \eo{} \\
\hline
Generics & The information generics provide may be embedded into objects as attributes. \\
Inheritance & Decorators may be used instead of inheritance. \\
Annotations & Annotation may be represented by decorators for the classes they annotate. \\
Lambda Functions & Functions may be implemented as no-attribute objects. \\
Reflection & Objects may have meta information embedded through regular attributes, accessible by other objects. \\
\ff{assert} & May be implemented as a normal validator with a special exception thrown in case of error. \\
Static Blocks & Each block may be represented by an object with. \\
Static Methods & Each static method may be an attribute in a class/factory. \\
\ff{finalize} & The object with \ff{finalize} may be decorated by a special decorator that receives a signal from garbage collector when it's time to destroy the object. \\
\ff{catch} & Each \ff{try/catch} block may be represented with a special object that validates the output of the object it decorates and if the returned value has no body, it does what the \ff{catch} block is supposed to do. \\
\ff{for}, \ff{while}, \ff{if}, \ff{switch}, \ff{break}, and \ff{continue} & All control-flow statements may be implemented as special objects in \eo{}, each of which is equivalent for the corresponding statement. \\
\hline
\end{tabularx}
\figcap{Most popular features in Java~8 and their possible representations in \eo{}.}
\label{tab:mapping}
\end{figure}

\subsection{Inheritance}

This Java code utilizes inheritance in order to reuse the functionality
provided by the parent class \ff{Shape} in the child class \ff{Circle}:

\begin{twocols}
\begin{ffcode}
abstract class Shape { |$\label{ln:java-shape}$|
  private float height;
  Shape(float h) {
    height = h;
  }
  float volume() {
    return square() * height;
  }
  abstract float square();
} |$\label{ln:java-shape-end}$|
final class Circle extends Shape { |$\label{ln:java-cicle}$|
  private float radius;
  Circle(float h, float r) {
    super(h);
    radius = r;
  }
  float square() {
    return 3.14 * radius * radius;
  }
}; |$\label{ln:java-circle-end}$|
\end{ffcode}
\end{twocols}

The method \ff{volume} relies on the functionality provided by the
abstract method \ff{square}, which is not implemented in the parent
class \ff{Shape}: this is why the class is declared as \ff{abstract}
and the method \ff{square} also has a modifier \ff{abstract}.
It's impossible to make an instance of the class \ff{Shape}. A child
class has to be define, which will inherit the functionality of
\ff{Shape} and implement the missing abstract method.

The class \ff{Cirle} does exactly that: it \ff{extends} the class
\ff{Shape} and implements the method \ff{square} with the functionality
that calculates the square of the circle using the radius.
The method \ff{volume} is present in the \ff{Circle} class, even
though it's implemented in the parent class.

This code would be represented in \eo{} as the following:

\begin{ffcode}
[child height] > shape
  [] > volume
    child.square.mul ^.height
[height radius] > circle |$\label{ln:eo-circle}$|
  shape $ height > @
  [] > square
    3.14.mul
      radius.mul
        radius |$\label{ln:eo-circle-end}$|
\end{ffcode}

There is not mechanism of inheritance in \eo{}, but decorating replaces
it with a slight modification of the structure of objects: the parent
object \ff{shape} has an additional attribute \ff{child}, which was
not explicitly present in Java. This attribute is the link to the object
that inherits \ff{shape}. Once the \ff{volume} is used, the attribute
refers to the child object and the functionality from \ff{circle} is used.
The same mechanism is implemented in Java ``under the hood'': \eo{}
makes it explicitly visible.

\subsection{Classes and Constructors}

There are no classes in \eo{} but only objects. Java, on the other hand,
is a class-oriented language. In the snippet
at the lines~\lrefs{java-shape}{java-circle-end}, \ff{Shape} is a class
and a better way of mapping it to \eo{} would be the following:

\begin{ffcode}
[] > shapes
  [c h] > new
    # Some extra functionality here, which
    # stays in the class constructor in Java
    []
      c > child
      h > height
      [] > volume
        child.square.mul ^.height
\end{ffcode}

Here, \ff{shapes} is the representation of Java class \ff{Shape}. It technically
is a factory of objects. In order to make a new, its attribute \ff{new}
must be used, which is very similar to the operator \ff{new} in Java.
The functionality of a Java constructor may also be implemented
in the attribute \ff{new}, such as a validation of inputs or
an initialization of local variables not passed through the constructor.

\subsection{Mutability}

All objects in \eo{} are immutable, which means that their attributes
can't be changed after an object is created. Java, on the other hand,
enables mutability. For example, both \ff{height} and \ff{radius} in the
lines~\lrefs{java-shape}{java-circle-end} are mutable attributes,
which can be modified after an object is instantiated. However,
the attribute \ff{radius} of the \eo{} object \ff{circle} at the
lines~\lrefs{eo-circle}{eo-circle-end} can't be modified. This
may be fixed by using the object \ff{memory}:

\begin{ffcode}
[height r] > circle
  memory r > radius
  shape $ height > @
  [] > square
    3.14.mul
      radius.mul
        radius
\end{ffcode}

An instance of the object \ff{memory} is created when the object \ff{circle}
is created, with the initial value of \ff{r}. Then, replacing the
object stored in the \ff{memory} is possible through its attribute \ff{write}:

\begin{ffcode}
circle 1.5 42.0 > c
c.radius.write 45.0
\end{ffcode}

This code makes an instance of \ff{circle} with the radius of \ff{42.0}.
Then, the radius is replaced with \ff{45.0}.

\subsection{Type Reflection}

There are no types in \eo{}, while Java not only have at least one type
for each object, but also enable the retrieval of this information in
runtime. For example, it's possible to detect the type of the shape
with this code:

\begin{ffcode}
if (s instanceof Circle) {
  System.out.println("It's a circle!");
}
\end{ffcode}

In \eo{} this meta-information about objects must be stored
explicitly in object attribute, in order to enable similar
reflection on types:

\begin{ffcode}
[height radius] > circle
  "circle" > type
  # The rest of the object
\end{ffcode}

Now, checking the type of the object is as easy as reading the
value of its attribute \ff{type}. The mechanism can be extended
with more additional information during the transition from
Java to \eo{}, such as information about attributes, decoratee, etc.

