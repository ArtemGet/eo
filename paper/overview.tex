The following four principles stay behind the
apparatus we introduce:

\begin{itemize}
\item An \emph{object} is a collection of \emph{attributes},
which are uniquely named bindings to objects. An object
is an \emph{atom} if its implementation is provided by the runtime.

\item An object is \emph{abstract} if at least one of its attributes
is \emph{free}---isn't \emph{bound} to any object. An object
is \emph{closed} otherwise.
\emph{Abstraction} is the process of creating an abstract object.
\emph{Application} is the process of making a \emph{copy} of an abstract
object, specifying some or all of its free attributes with
objects known as \emph{arguments}. Application may lead to the
creation of a closed object, or an abstract one, if not all free
attributes are specified with arguments.

\item An object may \emph{decorate} another object by binding it
to the $\varphi$ attribute of itself. A decorator has its
own attributes and the attributes of its decoratee.

\item A special attribute $\delta$ may be bound to \emph{data},
which is a computation platform dependable entity not
decomposable any further.
\emph{Dataization} is a process of retrieving data from an object,
by taking what the $\delta$ attribute is bound to.
The dataization of an object at the highest level of composition
leads to the execution of a program.
\end{itemize}

