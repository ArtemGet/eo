The following examples demonstrate a few Java programs and their alternatives
in \eo{}.

\subsection{Fibonacci Number}

Fibonacci sequence is a sequence of positive integers such that
each number is the sum of the two preceding ones, starting from $0$ and $1$, where:
\begin{equation*}
F_n = F_{n-1} + F_{n-2}.
\end{equation*}

The formula can be implemented in Java using recursion, as suggested
by~\citet[p.743]{deitel2007java} (code style is modified):

\begin{ffcode}
public class FibonacciCalculator {
  public long fibo(long n) {
    if (n < 2) {
      return n;
    } else {
      return fibo(n-1) + fibo(n-2);
    }
  }
}
\end{ffcode}

The same functionality would look in \eo{} like the following:

\begin{ffcode}
[n] > fibo
  if. > @
    n.lt 2
    n
    plus.
      fibo (n.minus 1)
      fibo (n.minus 2)
\end{ffcode}

\subsection{Determining Leap Year}

Consider a program to determine whether the year, provided
by the user as console input, is leap or not. The Java code,
as suggested by~\citet[pp.105--106]{liang2012}, would look like this
(the code style was slightly modified):

\begin{ffcode}
import java.util.Scanner;
public class LeapYear {
  public static void main(String[] args) {
    Scanner input = new Scanner(System.in);
    System.out.print("Enter a year: ");
    int year = input.nextInt();
    boolean isLeapYear =
      (year % 4 == 0 && year % 100 != 0) |\textbar\textbar|
      (year % 400 == 0);
    System.out.println(year +
      " is a leap year? " + isLeapYear);
  }
}
\end{ffcode}

The same functionality would require the following code in \eo{}:

\begin{ffcode}
+alias org.eolang.io.stdout
+alias org.eolang.io.stdin
+alias org.eolang.txt.sprintf

[args...] > main
  seq > @
    stdout
      "Enter a year:"
    stdout
      sprintf
        "%d is a leap year? %b"
        stdin.next.as-int > year
        or.
          and.
            eq. (mod. year 4) 0
            not. (eq. (mod. year 100) 0)
          eq. (mod. year 400) 0
\end{ffcode}

\subsection{Division by Zero}

As was explained by~\citet[p.314]{eckel2006thinking}, since division by zero
leads to a runtime exception, it is recommended to throw a more meaningful
exception to notify the user about the exceptional situation. This is how
it would be done in Java:

\begin{ffcode}
class Balance {
  // Calculate how much each user should
  // get, if we have this amount of users
  float share(int users) {
    if (users == 0) {
      throw new RuntimeException(
        "The number of users can't be zero"
      );
    }
    // Do the math and return the number
  }
}
\end{ffcode}

This is how this functionality would look in \eo{}:

\begin{ffcode}
[] > balance
  [users] > share
    if. > @
      eq. users 0
      []
        "The number can't be zero" > msg
        "InvalidInput" > type
      # Do the math and return
\end{ffcode}

If the \ff{users} argument is zero, an abstract object
will be returned, with a free body and two bound attributes
\ff{msg} and \ff{type}:

\begin{ffcode}
[]
  "The number of users can't be zero" > msg
  "InvalidInput" > type
\end{ffcode}

Once this object will be touched by the runtime, it will cause
the entire program to halt. This behavior is similar to what
is happening in Java with exceptions.

\subsection{Date Builder}

Creating a date/time object is a common task for most programs, which
is resolved in JDK8~\citep{jdk8} through the \ff{Calendar.Builder} class,
which suggests method cascading~\citep{beck1997smalltalk},
also known as fluent interface~\citep{fluentinterface}, for its users
(an innacurate and simplified example):

\begin{ffcode}
Calendar c = new Calendar.Builder()
  .setYear(2013)
  .setMonth(4)
  .setDay(6)
  .build();
\end{ffcode}

The implementation of an immutable version of the \ff{Calendar.Builder}
class would look like this in Java:

\begin{ffcode}
class Builder {
  private final int year;
  private final int month;
  private final int day;
  Builder(int y, int m, int d) {
    this.year = y;
    this.month = m;
    this.day = d;
  }
  Builder setYear(int y) {
    return new Builder(
      y, this.month, this.day
    );
  }
  Builder setMonth(int m) {
    return new Builder(
      this.year, m, this.day
    );
  }
  Builder setDay(int d) {
    return new Builder(
      this.year, this.month, d
    );
  }
  Calendar build() {
    return new Calendar(
      this.year, this.month, this.day
    );
  }
}
\end{ffcode}

This is how this functionality would look in \eo{}, combining
the builder and the calendar in one object:

\begin{ffcode}
[year month day] > calendar
  [y] > setYear
    calendar y month day > @
  [m] > setMonth
    calendar year m day > @
  [d] > setDay
    calendar year month d > @
  # The functionality of the calendar
  # goes in here...
\end{ffcode}

This is how it would be used in \eo{}:

\begin{ffcode}
calendar
.setYear 2013
.setMonth 4
.setDay 6
\end{ffcode}

\subsection{Streams}

Working with a flow of binary or text data requires the use
of stream objects, as explained by~\citet[p.226]{metsker2002}.
A non-canonical Java stream may be presented by
a two-methods interface and a sample implementation of it:

\begin{ffcode}
interface Stream {
  void print(String text);
  void close();
}
class ConsoleStream implements Stream {
  @Override
  void print(String text) {
    System.out.println(text);
  }
  @Override
  void close() {
    // Maybe something else
  }
}
\end{ffcode}

Then, it may be required to prepend all lines with a prefix. In order
to do this a decorator design pattern may be used,
as explained by~\citet[p.196]{gamma1994design}:

\begin{ffcode}
class PrefixedStream implements Stream {
  private final Stream origin;
  PrefixedStream(Stream s) {
    this.origin = s;
  }
  @Override
  void print(String text) {
    this.origin.print("DEBUG: " + text);
  }
  @Override
  void close() {
    this.origin.close();
  }
}
\end{ffcode}

The \ff{PrefixedStream} encapsulates an object of the same type it
implements. The decorator modifies the behavior of some methods (e.g., \ff{print()}), while
remain others untouched (e.g., \ff{close()}). This is how the same design
would look in \eo{}:

\begin{ffcode}
[] > console_stream
  [text] > print
    stdout > @
      text
  [] > close
    # Do something here
\end{ffcode}

Then, the decorator would look like this:

\begin{ffcode}
[@] > prefixed_stream
  [text] > print
    ^.@.print
      concat.
        "DEBUG: "
        text
\end{ffcode}

Here, the \ff{\^{}.@} attribute is the one that is being decorated.
The object \ff{prefixed\char`_stream} has attribute \ff{close} even
though it is not declared explicitly.

If the object is used like this, where \ff{stdout} is another stream,
printing texts to the console:

\begin{ffcode}
prefixed_stream(stdout)
  .print("Hello, world!")
\end{ffcode}

Then the console will print:

\begin{ffcode}
DEBUG: Hello, world!
\end{ffcode}

