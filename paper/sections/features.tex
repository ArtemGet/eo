\newcounter{feature}
\renewcommand\thefeature{\thesection.\arabic{feature}}
\newcommand\feature[2]{{\refstepcounter{feature}\label{feature:#1}\textbf{%
  %\thefeature.
  #2%
}.\;}}

There are a few features that distinguish \eo{} and \phic{}
from other existing OO languages and object theories, while some of
them are similar to what other languages have to offer. The Section is not
intended to present the features formally, which was done earlier in
Sections~\ref{sec:calculus} and~\ref{sec:syntax}, but to compare \eo{} with other
programming languages and informally identify similarities.

\feature{no-classes}{No Classes}
%
\eo{} is similar to other delegation-based languages like Self~\citep{ungar1987self},
where objects are not created by a class as in class-based languages like C++ or Java,
but from another object, inheriting properties from the original.
However, while in such languages, according to~\citet{fisher1995delegation},
``an object may be created,
and then have new methods added or existing methods redefined,''
in \eo{} such object alteration is not allowed.

\feature{no-types}{No Types}
%
Even though there are no types in \eo{}, compatibility
between objects may be inferred in compile-time and validated strictly, which other
\nospell{typeless} languages such as Python,
Julia~\citep{bezanson2012julia},
Lua~\citep{ierusalimschy2016lua},
or Erlang~\citep{erlang2020manual} can't guarantee.
Also, there is no type casting or reflection on types in \eo{}.

\feature{no-inheritance}{No Inheritance}
%
It is impossible to inherit attributes from another object in \eo{}.
The only two possible ways to re-use functionality are either via object
composition or decorators.
There are OO languages without implementation inheritance, for example Go~\citep{donovankernighan2015go},
but only Kotlin~\citep{jemerov2017kotlin} has decorators
as a language feature. In all other languages, the Decorator pattern~\citep{gamma1994design}
has to be implemented manually~\citep{bettinibono2008delegation}.

\feature{no-methods}{No Methods}
%
An object in \eo{} is a composition of other objects and atoms:
there are no methods or functions similar to Java or C++ ones.
Execution control is given to a program when atoms' attributes are referred to.
Atoms are implemented by \eo{} runtime similar to Java native objects.
To our knowledge, there are no other OO languages without methods.

\feature{no-ctors}{No Constructors}
%
Unlike Java or C++, \eo{} doesn't allow programmers to alter
the process of object construction or suggest alternative
paths of object instantiation via additional constructions.
Instead, all arguments are assigned to attributes ``as is'' and can't be modified.

\feature{no-static}{No Static Entities}
%
Unlike Java and C\#,
\eo{} objects may consist only of other objects, represented
by attributes, while class methods, also known as static methods, as well as
static literals, and static blocks---don't exist in \eo{}.
Considering modern programming languages, Go has no static methods either,
but only objects and ``\nospell{structs}''~\citep{schmager2010gohotdraw}.

\feature{no-primitives}{No Primitive Data Types}
%
There are no primitive data types in \eo{}, which
exist in Java and C++, for example.
As in Ruby, Smalltalk~\citep*{goldbergrobson1983smalltalk}, Squeak, Self, and Pharo,
integers, floating point numbers, boolean
values, and strings are objects in \eo{}:
``everything is an object'' is the key design principle, which,
according to~\citet[p.66]{west2004object}, is an ``obviously necessary prerequisite
to object thinking.''

\feature{no-operators}{No Operators}
%
There are no operators like \ff{+} or \ff{/} in \eo{}. Instead,
numeric objects have built-in atoms that represent math operations. The same
is true for all other manipulations with objects: they are provided
only by their encapsulated objects, not by external language constructs, as in
Java or C\#. Here \eo{} is similar to Ruby, Smalltalk and Eiffel,
where operators are syntax sugar, while implementation is encapsulated in
the objects.

\feature{no-null}{No NULL References}
%
Unlike C++ and Java, there is no concept of NULL in \eo{}, which
was called a ``billion dollar mistake'' by~\citet{hoare2009null} and
is one of the key threats for design consistency~\citep{eo1}.
Haskell, Rust, OCaml, Standard ML, and Swift also don't have NULL references.

\feature{no-empty}{No Empty Objects}
%
Unlike Java, C++ and all other OO languages,
empty objects with no attributes are forbidden in \eo{} in order
to guarantee the presence of object composition and
enable separation of concerns~\citep{dijkstra1982role}:
larger objects must always encapsulate smaller ones.

\feature{no-private}{No Private Attributes}
%
Similar to Python~\citep{lutz2013learning} and Smalltalk~\citep{hunt1997smalltalk},
\eo{} makes all object attributes publicly visible.
There are no protected ones, because there is no implementation
inheritance, which is considered harmful~\citep{hunt2000}.
There are no private attributes either, because information
hiding can anyway easily be violated via getters, and usually is, making the code longer
and less readable, as explained by~\citet{holub2004more}.

\feature{no-global}{No Global Scope}
%
All objects in \eo{} are assigned to some attributes. Objects constructed
in the global scope of visibility are assigned to attributes of the
$\Phi$ object of the highest level of abstraction.
Newspeak and Eiffel are two programming languages that does not have global scope as well.

\feature{no-mutability}{No Mutability}
%
Similar to Erlang~\citep{armstrong2010erlang},
there are only immutable objects in \eo{}, meaning that their attributes may
not be changed after the object is constructed or copied.
Java, C\#, and C++, have modifiers like
\ff{final}, \ff{readonly}, or \ff{const} to make attributes immutable, which
don't mean constants though. While the latter will
always expose the same functionality, the former may represent mutable
entities, being known as read-only references~\citep{birka2004practical}.
For example, an attribute \ff{r} may have an object \ff{random.pseudo}
assigned to it, which is a random number generator. \eo{} won't allow
assigning another object to the attribute \ff{r}. However, every time
the attribute is dataized, its value will be different.
%
There are number of OOP languages that also prioritize immutability of objects.
In Rust~\citep{matsakis2014rust}, for example, all variables are immutable by
default, but can be made mutable via the \ff{mut} modifier. Similarly,
D~\citep{bright2020origins} has qualifier \ff{immutable}, which expresses
transitive immutability of data.

\feature{no-exceptions}{No Exceptions}
%
In most OO languages exception handling~\citep{goodenough1975exception}:
happens through an imperative error-throwing statement.
Instead, \eo{} has a declarative mechanism for it, which
is similar to Null Object design pattern~\citep{martin1997pattern}: returning
an abstract object causes program execution to stop once the returned
object is dealt with.

\feature{no-functions}{No Functions}
%
There are no lambda objects or functions in \eo{}, which exist in Java~8+, for example.
However, objects in \eo{} have ``bodies,'' which make it possible to interpret
objects as functions.
Strictly speaking, if objects in \eo{} would only have bodies and no other attributes,
they would be functions. It is legit to
say that \eo{} extends lambda calculus, but in a different way
comparing to previous attempts made by~\citet{mitchell1993lambda} and \citet{di1998lambda}:
methods and attributes in \eo{} are not new concepts, but lower-level
objects.

\feature{no-mixins}{No mixins}
%
There are no ``traits'' or ``mixins'' in~\eo{}, which exist in Ruby and PHP to enable
code reuse from other objects without inheritance and composition.




