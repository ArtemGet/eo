In order to answer the question, whether
the proposed object calculus is sufficient
to express any object model, we are going to demonstrate
how four fundamental principles of OOP are realized by \phic{}:
encapsulation, abstraction, inheritance, and polymorphism.

\subsection{Abstraction}

Abstraction, which is called ``modularity'' by~\citet{grady2007object}, is,
according to~\citet[p.203]{west2004object},
``the act of separating characteristics into the relevant and
irrelevant to facilitate focusing on the relevant without
distraction or undue complexity.'' While \citet{stroustrup1997} suggests
C++ classes as instruments of abstraction, the ultimate goal
of abstraction is decomposition, according to~\citet[p.73]{west2004object}:
``composition is accomplished by applying abstraction---the
`knife' used to carve our domain into discrete objects.''

In \phic{} objects are the elements the problem domain
is decomposed into, which goes along the claim of~\citet[p.24]{west2004object}:
``objects, as abstractions of entities in the real world,
represent a particularly dense and cohesive clustering of information.''

\subsection{Inheritance}

Inheritance, according to~\citet{grady2007object}, is
``a relationship among classes wherein one class shares the structure
and/or behavior defined in one (single inheritance)
or more (multiple inheritance) other classes,'' where
``a subclass typically augments or
restricts the existing structure and behavior of its superclasses.''
The purpose of inheritance, according to~\citet{meyer1997object}, is
``to control the resulting potential complexity'' of the design
by enabling code reuse.

Consider a classic case of behaviour extension, suggested by~\citet[p.38]{stroustrup1997}
to illustrate inheritance. C++ class \ff{Shape} represents a graphic object
on the canvas (a simplified version of the original code):

\begin{ffcode}
class Shape {
  Point center;
public:
  void move(Point to) { center = to; draw(); }
  virtual void draw() = 0;
};
\end{ffcode}

The method \ff{draw()} is ``virtual,'' meaning that it is not implemented in the class
\ff{Shape} but may be implemented in sub-classes, for example in
the class \ff{Circle}:

\begin{ffcode}
class Circle : public Shape {
  int radius;
public:
  void draw() { /* To draw a circle */ }
};
\end{ffcode}

The class \ff{Circle} inherits the behavior of the class \ff{Shape} and
extends it with its own feature in the method \ff{draw()}. Now, when the
method \ff{Circle.move()} is called, its implementation from the class \ff{Shape}
will call the virtual method \ff{Shape.draw()}, and the call will be dispatched
to the overridden method \ff{Circle.draw()} through the ``virtual table'' in the class \ff{Shape}.
The creator of the class \ff{Shape} is now aware of sub-classes
which may be created long after, for example \ff{Triangle}, \ff{Rectangle}, and so on.

Even though implementation inheritance and method overriding seem to
be powerful mechanisms, they have been criticized.
According to~\citet{holub2003extends}, the main problem with
implementation inheritance is that it introduces unnecessary coupling
in the form of the ``fragile base class problem,''
as was also formally demonstrated by~\citet{mikhajlov1998study}.

The fragile base class problem is one of the reasons why
there is no implementation inheritance in \phic{}.
Nevertheless, object hierarchies to enable code reuse
in \phic{} may be created using decorators.
This mechanism is also known as ``delegation'' and, according
to~\citet[p.98]{grady2007object}, is ``an alternate approach to inheritance, in which
objects delegate their behavior to related objects.''
As noted by~\citet[p.139]{west2004object}, delegation is ``a way to extend or restrict
the behavior of objects by composition rather than by inheritance.''
\citet{seiter1998evolution} said that ``inheritance breaks encapsulation'' and suggested that
delegation, which they called ``dynamic inheritance,'' is a better way
to add behavior to an object, but not to override existing behavior.

The absence of inheritance mechanism in \phic{} doesn't make it
any weaker, since object hierarchies are available. \citet{grady2007object}
while naming four fundamental elements of object model mentioned
``abstraction, encapsulation, modularity, and hierarchy'' (instead of inheritance, like
some other authors).

\subsection{Polymorphism}

According to~\citet[p.467]{meyer1997object}, polymorphism means
``the ability to take several forms,'' specifically a variable
``at run time having the ability to become attached to objects
of different types, all controlled by the static declaration.''
\citet[p.67]{grady2007object} explains polymorphism as
an ability of a single name (such as a variable declaration)
``to denote objects of many different classes that are related by some common superclass,''
and calls it ``the most powerful feature of object-oriented programming languages.''

Consider an example C++ class, which is used by~\citet[p.310]{stroustrup1997}
to demonstrate polymorphism (the original code was simplified):

\begin{ffcode}
class Employee {
  string name;
public:
  Employee(const string& name);
  virtual void print() { cout << name; };
};
\end{ffcode}

Then, a sub-class of \ff{Employee} is created, overriding
the method \ff{print()} with its own implementation:

\begin{ffcode}
class Manager : public Employee {
  int level;
public:
  Employee(int lvl) :
    Employee(name), level(lvl);
  void print() {
    Employee::print();
    cout << lvl;
  };
};
\end{ffcode}

Now, it is possible to define a function, which accepts a set
of instances of class \ff{Employee} and prints them one by one,
calling their method \ff{print()}.:

\begin{ffcode}
void print_list(set<Employee*> &emps) {
  for (set<Employee*>::const_iterator p =
    emps.begin(); p != emps.end(); ++p) {
    (*p)->print();
  }
}
\end{ffcode}

The information of whether elements of the set \ff{emps} are instances of \ff{Employee}
or \ff{Manager} is not available for the \ff{print\char`\_list} function in compile-time.
As explained by \citet[p.103]{grady2007object},
``polymorphism and late binding go hand in hand;
in the presence of polymorphism, the binding of a method
to a name is not determined until execution.''

Even though there are no explicitly defined types in \phic{},
the comformance between objects is derived and ``strongly'' checked
in compile time. In the example above, it would not be possible to
compile the code that adds elements to the set \ff{emps}, if any
of them lacks the attribute \ff{print}. Since in \eolang{},
there is no reflection on types or any other mechanisms
of alternative object instantiation, it is always known where
objects are constructed or copied and what is the structure of them.
Having this information in compile-time it is possible to guarantee
strong compliance of all objects and their users. To our knowledge,
this feature is not available in any other OOP languages.

\subsection{Encapsulation}

Encapsulation is considered the most important principle of OOP
and, according to~\citet[p.51]{grady2007object},
``is most often achieved through information hiding,
which is the process of hiding all the secrets of an object
that do not contribute to its essential characteristics;
typically, the structure of an object is hidden, as well
as the implementation of its methods.'' Encapsulation in C++ and
Java is achieved through access modifiers like \ff{public} or
\ff{protected}, while in some other languages, like JavaScript or Python,
there are no mechanisms of enforcing information hiding.

However, even though~\citet[p.51]{grady2007object} believe that
``encapsulation provides explicit barriers among different
abstractions and thus leads to a clear separation of concerns,''
in reality the barriers are not so explicit: they can be easily
violated.
\citet[p.141]{west2004object} noted that
``in most ways, encapsulation is a discipline more than a real barrier;
seldom is the integrity of an object protected in any absolute
sense, and this is especially true of software objects,
so it is up to the user of an object to respect that object's encapsulation.''
There are even programming ``best practices,'' which encourage
programmers to compromise encapsulation: getters and setters are
the most notable example, as was demonstrated by~\citet{holub2004more}.

The inability to make the encapsulation barrier explicit
is one of the main reasons why there is no information hiding in \phic{}.
Instead, all attributes of all objects in \phic{}
are visible to any other object.

In \eolang{} the primary goal of encapsulation is achieved differently.
The goal is to reduce coupling between objects:
the less they know about each other the thinner the
the connection between them, which is one of the virtues of
software design, according to~\citet{yourdon1979structured}.

In \eolang{} the tightness of coupling between objects should be controlled
during the build, similar to how the threshold of test code coverage is
usually controlled. At compile-time the compiler collects the information
about the relationships between objects and calculates the coupling depth of each connection.
For example, the object \ff{garage} is referring to the object
\ff{car.engine.size}. This means that the depth of this connection between objects
\ff{garage} and \ff{car} is two, because the object \ff{garage} is using
two dots to access the object \ff{size}. Then, all collected depths from
all object connections are analyzes and the build is rejected if the numbers
are higher than a predefined threshold. How exactly the numbers are analyzed
and what are the possible values of the threshold is a subject for future
researches.



