First, the source code of \eolang{} is parsed by ANTLR4-powered
parser and an intermediate representation is built in XML,
as was demonstrated in \cref{sec:xml}.
The output format is called \xmir{}.
One \ff{.eo} file with
the source code in \eolang{} produces one \xmir{} file with \ff{.xml} extension.

\xmir{} is then refined via a \emph{pipeline} of XSLT stylesheets.
For example, \xmir{} at \lrefs{xml-circle}{xml-circle-end} contains a
reference to the object \ff{r} at \lref{xml-circle-r}.
An XSL transformation adds an attribute \ff{ref} to the XML element \ff{<o/>},
referring it to the line inside XML document, where the object \ff{r} is defined:

\begin{ffcode}
<o name="circle">
  <o name="r"/> (*@\label{ln:xml-circle2-r}@*) <!-- (*@\texttt{\lref{xml-circle2-r}}@*) -->
  <o base=".times" name="square">
    <o base="r" (*@\textbf{\texttt{ref="\lref{xml-circle2-r}"}}@*)/>
    <o base="int" data="int">2</o>
    <o base="float" data="float">3.14</o>
  </o>
</o>
\end{ffcode}

There are over two dozens XSL transformations in the pipeline, which
are applied to the \xmir{} in a specific order. New transformations can
be added to the pipeline for example in order to detect inconsistencies
in \xmir{}, enforce new semantic rules, or optimize object structures.

Then, \xmir{} can be translated to machine code, bytecode, C++ source code,
or any other target platform language. We implemented
a translator to Java source code, which represents
\xmir{} objects as Java classes and attributes as pairs in encapsulated
\ff{java.util.HashMap} instances.

Then, Java source code is compiled to bytecode by OpenJDK Java compiler.
Then, runtime dependencies with atoms are taken from Maven Central
and placed to the Java classpath.

Finally, JRE runs the program through \ff{Main.java} class together with
all \ff{.jar} dependencies in the classpath.

