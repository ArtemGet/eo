Fig.~\ref{fig:bnf} demonstrates the entire syntax of \eo{} language in BNF.
Similar to Python~\citep{lutz2013learning}, indentation in \eo{} is part of the syntax:
the scope of a code block is determined by its horizontal position
in relation to other blocks, which is also known as ``off-side rule''~\citep{landin1966next}.

There are no keywords in \eo{} but only a few special symbols
denoting grammar constructs:
  \ff{>} for the attributes naming,
  \ff{.} for the dot notation,
  \ff{[]} for the specification of parameters of abstract objects,
  \ff{()} for scope limitations,
  \ff{!} for turning objects into constants,
  \ff{:} for naming arguments,
  \ff{"} (double quotes) for string literals,
  \ff{@} for the \nospell{decoratee},
  \ff{'} (apostroph) for explicit copying,
  \ff{\^{}} for referring to the parent object,
  and
  \ff{\$} for referring to the current object.
Attributes, which are the only identifiers that exist in \eo{}, may have
any Unicode symbols in their names, as long as they start with a small English letter
and don't contain spaces, line breaks, or special symbols mentioned above:
\ff{test-File} and
\begin{CJK}{UTF8}{gbsn}
\ff{i文件}
\end{CJK}
 are valid identifiers.
Identifiers are case-sentitive: \ff{car} and \ff{Car} are two different identifiers.
Java-notation is used for numbers and strings.

\newcommand\sntx[1]{{\sffamily #1}}
\begin{figure*}
\raggedright
\small
\setlength\columnsep{24pt}
\newcommand\T[1]{\sntx{#1}}
\newcommand\V[1]{{\ttfamily `#1'}}
\newcommand\RE[1]{{\ttfamily \textcolor{gray}{/}#1\textcolor{gray}{/}}}
\newcommand\alt{ \textcolor{gray}{|} }
\newcommand\grp[1]{\textcolor{gray}{(} #1 \textcolor{gray}{)}}
\newcommand\opt[1]{\textcolor{gray}{[} #1 \textcolor{gray}{]}}
\newcommand\few[1]{\textcolor{gray}{\{} #1 \textcolor{gray}{\}}}
\newcommand\lex[1]{{\scshape\textcolor{darkgray}{#1}}}
\newcommand\df{\>{\ttfamily\textcolor{gray}{::=} }}
\begin{multicols}{2}
\begin{tabbing}
\hspace*{1.6cm}\= \kill

\T{program} \df \opt{\T{license}} \opt{\T{metas}} \T{objects} \\
\T{objects} \df \T{object} \lex{eol} \few{\T{object} \lex{eol}} \\
\T{license} \df \few{\T{comment}} \lex{eol} \\
\T{metas} \df \few{\T{meta}} \lex{eol} \\
\T{comment} \df \V{\#} \few{\lex{any}} \lex{eol} \\
\T{meta} \df \V{+} \T{name} \opt{\V{␣} \lex{any} \few{\lex{any}}} \lex{eol} \\
\T{name} \df \RE{[a-z]} \few{\lex{any}} \\
\T{object} \df \grp{ \T{foreign} \alt \T{local} } \\
\T{foreign} \df \T{abstraction} \V{␣/} \T{name} \alt \V{?} \\
\T{local} \df \grp{ \T{abstraction} \alt \T{application} } \T{details} \\
\T{details} \df \opt{\T{tail}} \few{\T{vtail}} \\
\T{tail} \df \lex{eol} \lex{tab} \few{\T{object} \lex{eol}} \lex{untab} \\
\T{vtail} \df \lex{eol} \T{ref} \opt{\T{htail}} \opt{\T{suffix}} \opt{\T{tail}} \\
\T{abstraction} \df \T{attributes} \opt{ \T{suffix} } \\
\T{attributes} \df \V{[} \T{attribute} \few{\V{␣} \T{attribute}} \V{]} \\
\T{attribute} \df \V{@} \alt \T{name} \opt{\V{...}} \\
\T{suffix} \df \V{␣} \V{>} \V{␣} \grp{\V{@} \alt \T{name}} \opt{\V{!}} \\
\T{ref} \df \V{.} \grp{ \T{name} \alt \V{\^{}} \alt \V{@} \alt \V{<} } \\
\T{application} \df \T{head} \opt{\T{htail}} \\
\T{htail} \df \T{application} \T{ref} \\
  \> \alt \V{(} \T{application} \V{)} \\
  \> \alt \T{application} \V{:} \T{name} \\
  \> \alt \T{application} \T{suffix} \\
  \> \alt \T{application} \V{␣} \T{application} \\
\T{head} \df \opt{\V{...}} \grp{ \T{name} \opt{\V{'}} \alt \T{data} \alt \V{@} \alt \V{\$} \alt \V{\&} \\
  \> \alt \V{\^{}} \alt \V{*} \alt \T{name} \V{.} \alt \V{Q} \alt \V{QQ} } \\
\T{data} \df \T{bytes} \alt \T{bool} \alt \T{string} \alt \T{integer} \alt \T{float} \\
\T{bytes} \df \T{byte} \few{\V{-} \T{byte}} \alt \V{-{}-} \\
\T{bool} \df \V{TRUE} \alt \V{FALSE} \\
\T{byte} \df \RE{[\textbackslash{}dA-F][\textbackslash{}dA-F]} \\
\T{string} \df \RE{"[\^{}"]*"} \\
\T{integer} \df \RE{[+-]?\textbackslash{}d+|0x[a-f\textbackslash{}d]+} \\
\T{float} \df \RE{[+-]?\textbackslash{}d+(\textbackslash{}.\textbackslash{}d+)?} \opt{ \RE{e(+|-)?\textbackslash{}d+} } \\
\end{tabbing}
\end{multicols}
\figcap{The full syntax of \eo{} in BNF.
  \lex{eol} is a line ending that preserves the indentation of the previous line.
  \lex{tab} is a right-shift of the indentation, while \lex{untab} is a left-shift.
  \lex{any} is any symbol excluding \lex{eol}, \V{.}, and \V{␣}.
  The texts between forward slashes are Perl-style regular expressions.}
\label{fig:bnf}
\end{figure*}

\subsection{Identity, State, and Behavior}

According to~\citet{grady2007object}, an object in OOP has state, behavior, and identity:
``The state of an object encompasses all of the properties of
the object plus the current values of each of these properties.
Behavior is how an object acts and reacts, in terms of its state changes and message passing.
Identity is that property of an object which distinguishes it from all other objects.''
The syntax of \eo{} makes a difference between these three categories.

This is a declaration of an abstract object \ff{book},
which has a single \emph{identity} attribute \ff{isbn}:

\begin{ffcode}
[isbn] > book |$\label{ln:book}$|
\end{ffcode}

To make a new object with a specific ISBN, the \ff{book}
has to be \emph{copied}, with the \emph{data} as an argument:

\begin{ffcode}
book "978-1519166913" > b1
\end{ffcode}

Here, \ff{b1} is a new object created.
Its only attribute is accessible as \ff{b1.isbn}.

A similar abstract object, but with two new \emph{state} attributes, would
look like:

\begin{ffcode}
[isbn] > book2  |$\label{ln:book2}$|
  "Object Thinking" > title
  memory 0 > price |$\label{ln:book2-end}$|
\end{ffcode}

The attribute \ff{title} is a constant, while the \ff{price}
represents a mutable chunk of bytes in computing memory. They both are
accessible similar to the \ff{isbn}, via \ff{book2.title}
and \ff{book2.price}. It is legal to access them in the abstract
object, since they are bound to objects. However, accessing \ff{book2.isbn}
will lead to an error, since the attribute \ff{isbn} is free
in the abstract object \ff{book2}.

A \emph{behavior} may be added to an object with a new \emph{inner}
abstract object \ff{set-price}:

\begin{ffcode}
[isbn] > book3 |$\label{ln:book3}$|
  "Object Thinking" > title
  memory 0 > price
  [p] > set-price
    ^.price.write p > @ |$\label{ln:book3-end}$|
\end{ffcode}

The price of the book may be changed with this one-liner:

\begin{ffcode}
book3.set-price 19.99
\end{ffcode}

\subsection{Indentation}

This is an example of an abstract object \ff{vector}, where
spaces are replaced with the ``\textvisiblespace'' symbol in order to demonstrate
the importance of their presence in specific quantity
(for example, there has to be exactly one space after the closing square bracket at the
second line and the \ff{>} symbol, while two spaces will break the syntax):

\begin{ffcode*}{showspaces=true}
# This is a vector in 2D space |$\label{ln:comment}$|
[dx dy] > vector |$\label{ln:vector}$|
  sqrt. > length |$\label{ln:length}$|
    plus.
      dx.times dx
      dy.timex dy |$\label{ln:length-end}$| |$\label{ln:vector-end}$|
\end{ffcode*}

The code at \lref{comment} is a \emph{comment}.
Two \emph{free attributes} \ff{dx} and \ff{dy}
are listed in square brackets at \lref{vector}.
The \emph{name} of the object goes after the \ff{>} symbol.
The code at \lref{length} defines
a \emph{bound attribute} \ff{length}. Anywhere when an object
has to get a name, the \ff{>} symbol can be added after the object.

The declaration of the attribute \ff{length} at \lrefs{length}{length-end}
can be written in one line, using \emph{dot notation}:

\begin{ffcode}
((dx.times dx).plus (dy.timex dy)).sqrt > length
\end{ffcode}

An \emph{inverse} dot notation is used in order to simplify
the syntax. The identifier that goes after the dot is written
first, the dot follows, and the next line contains the part
that is supposed to stay before the dot. It is also possible to rewrite
this expression in multiple lines without the usage of
inverse notation, but it will look less readable:

\begin{ffcode}
dx.times dx |$\label{ln:dx-pow}$|
.plus
  dy.timex dy |$\label{ln:dx-pow-2}$|
.sqrt > length |$\label{ln:dx-pow-3}$|
\end{ffcode}

Here, \lref{dx-pow} is the application of the object \ff{dx.times} with
a new argument \ff{dx}. Then, the next line is the object \ff{plus} taken
from the object created at the first line, using the dot notation. Then,
\lref{dx-pow-2} is the argument passed to the object \ff{plus}.
The code at \lref{dx-pow-3} takes the object \ff{sqrt} from the object constructed
at the previous line, and gives it the name \ff{length}.

Indentation is used for two purposes: either to define attributes
of an abstract object or to specify arguments for object application, also
known as making a \emph{copy}.
A definition of an abstract object starts with a list of free attributes
in square brackets on one line, followed by a list of bound attributes
each in its own line. For example, this is an abstract \emph{anonymous} object
(it doesn't have a name)
with one free attribute \ff{x} and two bound attributes \ff{succ} and \ff{prev}:

\begin{ffcode}
[x]
  x.plus 1 > succ
  x.minus 1 > prev
\end{ffcode}

The arguments of \ff{plus} and \ff{minus} are provided in a \emph{horizontal}
mode, without the use of indentation. It is possible to rewrite this code
in a \emph{vertical} mode, where indentation will be required:

\begin{ffcode}
[x] |$\label{ln:succ}$|
  x.plus > succ
    1
  x.minus > prev
    1 |$\label{ln:succ-end}$|
\end{ffcode}

This abstract object can also be written in a horizontal mode,
because it is anonymous:

\begin{ffcode}
[x] (x.plus 1 > succ) (x.minus 1 > prev)
\end{ffcode}

\subsection{\eo{} to XML}\label{sec:xml}

Due the nesting nature of \eo{}, its program can be transformed
to an XML document. The abstract object \ff{vector} would produce
this XML tree of elements and attributes:

\begin{ffcode}
<o name="vector">
  <o name="dx"/>
  <o name="dy"/>
  <o name="length" base=".sqrt">
    <o base=".plus">
      <o base=".times">
        <o base="dx"/>
        <o base="dx"/>
      </o>
      <o base=".times">
        <o base="dy"/>
        <o base="dy"/>
      </o>
    </o>
  </o>
</o>
\end{ffcode}

Each object is represented by an \ff{<o/>} XML element with a few
optional attributes, such as \ff{name} and \ff{base}. Each
attribute is either a named reference to an object (if the attribute is bound,
such as \ff{length}), or a name without a reference (if it is free,
such as \ff{dx} and \ff{dy}).

\subsection{Data Objects and Arrays}

There are a few abstract objects which can't be directly copied, such as
\ff{float} and \ff{int}. They are created by the compiler when it meets
a special syntax for data, for example:

\begin{ffcode}
[r] > circle
  r.times 3.14 > circumference
\end{ffcode}

This syntax would be translated to XMIR (XML based
data format explained in Sec.~\ref{sec:xmir}):

\begin{ffcode}
<o name="circle"> |$\label{ln:xml-circle}$|
  <o name="r"/>
  <o base=".times" name="circumference">
    <o base="r"/>  |$\label{ln:xml-circle-r}$|
    <o base="float" data="float">3.14</o>
  </o>
</o> |$\label{ln:xml-circle-end}$|
\end{ffcode}

Each object, if it is not abstract, has a ``base'' attribute in XML,
which contains that name of an abstract object to be copied. The
object \ff{float} is abstract, but its name
can't be used directly in a program, similar to how \ff{r} or \ff{times}
are used. The only way to make a copy of the abstract object \ff{float}
is to use a numeric literal like \ff{3.14}.

The abstract objects which can't be used directly and can only be
created by the compiler through \sntx{data}---are called \emph{data}.
The examples of some possible data are in the Tab.~\ref{tab:types}.

\begin{table}
\begin{tabularx}{\columnwidth}{l|X|r}
\toprule
Data & Example & Size \\
\midrule
\ff{bytes} & \ff{1F-E5-77-A6} & 4 \\
\ff{string} & \ff{"Hello, \foreignlanguage{russian}{друг}!"} & 16 \\
  & \ff{"\textbackslash{}u5BB6"} or \begin{CJK}{UTF8}{gbsn}\ff{"家"}\end{CJK} & 2 \\
\ff{int} & \ff{1024}, \ff{0x1A7E}, or \ff{-42} & 8 \\
\ff{float} & \ff{3.1415926} or \ff{2.4e-34} & 8 \\
\ff{bool} & \ff{TRUE} or \ff{FALSE} & 1 \\
\bottomrule
\end{tabularx}
\figcap{The syntax of all data with examples. The ``Size'' column
denotes the number of bytes in the \ff{as-bytes} attribute.
UTF-8 is the encoding used in \ff{string} object.}
\label{tab:types}
\end{table}

The \ff{array} is yet another data, which can't be copied
directly. There is a special syntax for making arrays,
which looks similar to object copying:

\begin{ffcode}
* "Lucy" "Jeff" 3.14 |$\label{ln:array-1}$|
* > a |$\label{ln:array-2a}$|
  (* "a")
  TRUE |$\label{ln:array-2b}$|
* > b |$\label{ln:array-3}$|
\end{ffcode}

The code at \lref{array-1} makes an array of three elements: two strings
and one float. The code at \lrefs{array-2a}{array-2b} makes an array named \ff{a} with another
array as its first element and \ff{TRUE} as the second item.
The code at \lref{array-3} is an empty array with the name \ff{b}.

\subsection{Varargs}

The last free attribute in an abstract object may be a \emph{vararg},
meaning that any number or zero arguments may be provided. All of them
will be packaged in an array by the compiler, for example:

\begin{ffcode}
[x...] > sum |$\label{ln:sum-def}$|
sum 8 13 -9 |$\label{ln:sum-instance}$|
\end{ffcode}

Here, at the first line the abstract object \ff{sum} is defined
with a free vararg attribute \ff{x}. At the second line a copy of the
abstract object is made with three arguments. The internals of
the \ff{sum} will see \ff{x} as an \ff{array} with three
elements inside.

It is possible to provide an array as a parameter for vararg attribute
(both variants are semantically equivalent):

\begin{ffcode}
* 42 7 > a
sum ...a
sum 42 7
\end{ffcode}

\subsection{Scope Brackets}

Brackets can be used to group object arguments in horizontal mode:

\begin{ffcode}
sum (div 45 5) 10  |$\label{ln:sum}$|
\end{ffcode}

The \ff{(div 45 5)} is a copy of the abstract object \ff{div}
with two arguments \ff{45} and \ff{5}. This object is itself
the first argument of the copy of the object \ff{sum}. Its second
argument is \ff{10}. Without brackets the syntax would read differently:

\begin{ffcode}
sum div 45 5 10
\end{ffcode}

This expression denotes a copy of \ff{sum} with four arguments.

\subsection{Inner Objects}

An abstract object may have other abstract objects as its attributes,
for example:

\begin{ffcode}
# A point on a 2D canvas
[x y] > point
  [to] > distance
    length. > len |$\label{ln:vector-length}$|
      vector
        to.x.minus (^.x)
        to.y.minus (^.y)
\end{ffcode}

The object \ff{point} has two free attributes \ff{x} and \ff{y}
and the attribute \ff{distance}, which is bound to an abstract
object with one free attribute \ff{to} and one bound attribute \ff{len}.
The \emph{inner} abstract object \ff{distance} may only be copied
with a reference to its \emph{parent} object \ff{point}, via
a special attribute denoted by the \ff{\^{}} symbol:

\begin{ffcode}
distance. |$\label{ln:point-copy}$|
  point
    5:x
    -3:y
  point:to
    13:x
    3.9:y
\end{ffcode}

The parent object is \ff{(point 5 -3)}, while the object \ff{(point 13 3.9)}
is the argument for the free attribute \ff{to} of the object \ff{distance}.
Suffixes \ff{:x}, \ff{:y}, and \ff{:to} are optional and may be used
to denote the exact name of the free attribute to be bound to the
provided argument.

\subsection{Decorators}

An object may extend another object by \emph{decorating} it:

\begin{ffcode}
[center radius] > circle |$\label{ln:circle}$|
  center > @ |$\label{ln:circle-phi}$|
  [p] > is-inside
    lte. > @
      ^.@.distance $.p  |$\label{ln:circle-phi-2}$|
      ^.radius |$\label{ln:circle-end}$|
\end{ffcode}

The object \ff{circle} has a special attribute \ff{@}
at \lref{circle-phi}, which denotes
the \emph{decoratee}: an object to be extended,
also referred to as ``component'' by~\citet{gamma1994design}.

The \emph{decorator} \ff{circle}
has the same attributes as its decoratee \ff{center}, but also
its own attribute \ff{is-inside}. The attribute \ff{@} may be used
the same way as other attributes, including in dot notation, as it is done
at \lref{circle-phi-2}. However, this line
may be re-written in a more compact way, omitting the explicit
reference to the \ff{@} attribute, because all attributes
of the \ff{center} are present in the \ff{circle};
and omitting the reference to \ff{\$} because the default scope of visibility of
\ff{p} is the object \ff{is-inside}:

\begin{ffcode}
^.distance p
\end{ffcode}

The inner object \ff{is-inside} also has the \ff{@} attribute: it
decorates the object \ff{lte} (stands for ``less than equal'').
The expression at \lref{circle-phi-2} means:
take the parent object of \ff{is-inside},
take the attribute \ff{@} from it, then take the inner object \ff{distance}
from there, and then make a copy of it with the attribute \ff{p}
taken from the current object (denoted by the \ff{\$} symbol).

The object \ff{circle} may be used like this, to understand whether
the $(0,0)$ point is inside the circle at $(-3,9)$ with the radius $40$:

\begin{ffcode}
circle (point -3 9) 40 > c  |$\label{ln:circle-c}$|
c.is-inside (point 0 0) > i
\end{ffcode}

Here, \ff{i} will be a copy of \ff{bool} behaving like \ff{TRUE}
because \ff{lte} decorates \ff{bool}.

It is also possible to make decoratee free, similar to other free
attributes, specifying it in the list of free attributes in
square brackets.

\subsection{Anonymous Abstract Objects}

An abstract object may be used as an argument of another object while
making a copy of it, for example:

\begin{ffcode}
(dir "/tmp").walk
  *
    [f]
      f.is-dir > @
\end{ffcode}

Here the object \ff{walk} is copied with a single argument:
the one-item array, which is an
abstract object with a single free attribute \ff{f}. The \ff{walk}
will use this abstract object, which doesn't have a name, in order
to filter out files while traversing the tree of directories. It will
make a copy of the abstract object and then treat it as a boolean
value in order to make a decision about each file.

The syntax may be simplified and the abstract object may be inlined
(the array is also inlined):

\begin{ffcode}
(dir "/tmp").walk
  * ([f] (f.is-dir > @))
\end{ffcode}

An anonymous abstract object may have multiple attributes:

\begin{ffcode}
[x] (x.plus 1 > succ) (x.minus 1 > prev)
\end{ffcode}

This object has two attributes \ff{succ} and \ff{prev}, and doesn't
have a name.

The parent of each copy of the abstract object will be set by
the object \ff{walk} and will point to the \ff{walk} object itself.

\subsection{Constants}

\eo{} is a declarative language with lazy evaluations. This means
that this code would read the input stream two times:

\begin{ffcode}
[] > hello
  stdout > say
    sprintf
      "The length of %s is %d"
      stdin.next-line > x!
      x.length
\end{ffcode}

The \ff{sprintf} object will go to the \ff{x} two times. First time,
in order to use it as a substitute for \ff{\%s} and the second time for
\ff{\%d}. There will be two round-trips to the standard input stream, which
obviously is not correct. The exclamation mark at the \ff{x!} solves the
problem, making the object by the name \ff{x} a \emph{constant}. This means
that all attributes of \ff{x} are \emph{cached}. Important to notice
that the cache is \emph{not deep}: the attributes of attributes are not cached.

Here, \ff{x} is an attribute of the object \ff{hello}, even though
it is not defined as explicitly as \ff{say}. Anywhere a new
name shows up after the \ff{>} symbol, it is a declaration of a new
attribute in the nearest object abstraction.

\subsection{Explicit Shallow Copies}

There may be a need to make a copy of an object without giving any
parameters to it. This may be done with an apostrophe suffix:

\begin{ffcode}
point' > p
p 3 5 > p1
\end{ffcode}

Here, two objects will be created, \ff{p} and \ff{p1}, where
the former is an abstract one, a copy of \ff{copy}, while the
later is a closed one with two parameters specified. The
apostrophe suffix may be used anywhere after the name of an object,
for example:

\begin{ffcode}
circle
  point' 3 a'
\end{ffcode}

Making a copy of \ff{circle} will not lead to making a copy of \ff{point},
which is encapsulated by \ff{circle}.

\subsection{Object Identity}

Each object has a special attribute \ff{<}, which is an integer
refering to a unique identifier of an object in the entire
runtime scope of a program. All of the following expressions are true:

\begin{ffcode}
TRUE.<.eq (TRUE.<)
42.<.eq (42.<)
point.<.eq (point.<)
\end{ffcode}

All of the following expressions are false:

\begin{ffcode}
42.<.eq (7.<)
(2.plus 2).<.eq (4.<)
(point 3 5).<.eq ((point 3 5).<)
(* 1 2).<.eq ((* 1 2).<)
\end{ffcode}

\subsection{Metas and License}

A program may have a comment at the beginning of the file, which
is called a \emph{license}. The license may be followed by an optional
list of \emph{meta} statements, which are passed to the compiler
as is. The meaning of them depends on the compiler and may vary
between target platforms. This program instructs the compiler
to put all objects from the file into the package \ff{org.example}
and helps it resolve the name \ff{stdout}, which is external
to the file:

\begin{ffcode}
# (c) John Doe, 2021
# All rights reserved.
# The license is MIT

+package org.example
+alias org.eolang.io.stdout

[args...] > app
  stdout > @
    "Hello, world!\n"
\end{ffcode}

\subsection{Atoms}

Some objects in \eo{} programs may need to be platform specific
and can't be composed from other existing objects---they are called
\emph{atoms}.
For example, the object \ff{app} uses the object \ff{stdout},
which is an atom. Its implementation would be provided by the
runtime. This is how the object may be defined:

\begin{ffcode}
+rt jvm org.eolang:eo-runtime:0.7.0
+rt ruby eolang:0.1.0

[text] > stdout /bool |$\label{ln:stdout}$|
\end{ffcode}

The \ff{/bool} suffix informs the compiler that this object must
not be compiled from \eo{} to the target language. The object
with this suffix already exists in the target language and most
probably could be found in the library specified by the \ff{rt}
meta. The exact library to import has to be selected by the compiler.
In the example above, there are two libraries specified: for JVM and
for Ruby.

The \ff{bool} part after the \ff{/} is the name of
object, which \ff{stdout} decorates. The name may be replaced by
a question mark, if uncertain about the object being decorated.

Atoms in \eo{} are similar to ``native'' methods in Java and ``\nospell{extern}'' methods in C\#: this mechanism is also known as foreign function interface (FFI).

\subsection{Home Object}

An instance of an abstact object may need to have access to the
object where the abstract was defined, for example this is
how object \ff{array.map} is implemented in Objectionary:

\begin{ffcode}
[] > list |$\label{ln:list-parent}$|
  [f] > mapi /list |$\label{ln:list-map}$|
  [f] > map
    ^.mapi > @
      [x i] |$\label{ln:map-inner}$|
        &.f x > @
\end{ffcode}

The object \ff{mapi} at \lref{list-map} is an atom:
it iterates through list of items and makes
a copy of the provided two-arguments abstract object \ff{f} applying the next
item to it and the index in the array (that is why the name with the ``i'' suffix).

The object \ff{map} does exactly the same, but doesn't provide the
index of each element to the inner abstract object.
The anonymous inner abstract object at \lref{map-inner}
has to get access to the attribute \ff{f} of \ff{map}.
However, \ff{\^{}.f} won't work, because the parent of it is a copy of \ff{mapi}, and the
parent of \ff{mapi} is the object \ff{list}.
Thus, there is no way to get access
to \ff{map.f} using parent attributes.

The \emph{home} attribute \ff{\&} helps here.
Once an abstract object at \lref{map-inner} is created, its
home attribute is set to the abstract object \ff{list} at \lref{list-parent}.
Its parent attribute \ff{\^{}} is also set to the object \ff{list},
but is later changed by the atom \ff{mapi} when a copy of it is being made.
However, the home attribute remains the same.
