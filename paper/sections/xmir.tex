Consider this sample \eolang{} snippet:

\begin{ffcode}
[x y] > app
  42 > n!
  [t] > read
    minus.
      n.neg
      t
\end{ffcode}

It will be compiled to the following XMIR:

\begin{ffcode}
<o line="1" name="app">
  <o line="1" name="x"/>
  <o line="1" name="y"/>
  <o base="int" const="" data="int"
    line="2" name="n">42</o>   (*@\label{ln:xml-data}@*)
  <o line="3" name="read">
    <o line="3" name="t"/>
    <o base=".minus" line="4"> (*@\label{ln:xml-minus}@*)
      <o base="n" line="5"/> (*@\label{ln:method-start}@*)
      <o base=".neg" line="5" method=""/>  (*@\label{ln:method-end}@*)
      <o base="t" line="6"/>
    </o> (*@\label{ln:xml-minus-end}@*)
  </o>
</o>
\end{ffcode}

Each object is translated to XML element \ff{<o>}, which has
a number of optional attributes and a mandatory one: \ff{line}.
The attribute \ff{line} always contains a number of the
line where this object was seen in the source code.

The attribute \ff{name} contains the name of the object, if
it was specified with the \ff{>} syntax construct.

The attribute \ff{base} contains the name of the object, which
is being copied. If the name starts with a dot, this means
that it refers to the first \ff{<o>} child.

The attribute \ff{method} may be temporarily present, suggesting
further steps of XMIR transformations to modify the code
at \lrefs{xml-minus}{xml-minus-end} to the following (the attribute \ff{method}
goes away):

\begin{ffcode}
<o base=".minus" line="4"> (*@\label{ln:new-minus}@*)
  <o base=".neg" line="5"/> (*@\label{ln:xml-neg}@*)
    <o base="n" line="5"/>
  </o> (*@\label{ln:xml-neg-end}@*)
  <o base="t" line="6"/>
</o> (*@\label{ln:new-minus-end}@*)
\end{ffcode}

The semantics of the element \ff{.minus} at~\lref{xml-minus}
and the element \ff{.neg} at~\lref{xml-neg} are similar: prefix
notation for child-parent relationships. The XML element at~\lrefs{xml-neg}{xml-neg-end}
means \ff{neg. n} or \ff{n.neg}. The
leading dot at the attribute \ff{base} means that the first
child of this XML element is the \eolang{} parent of it.
Similarly, the XML element at ~\lrefs{new-minus}{new-minus-end}
means \ff{minus. (neg. n) t}, or \ff{(neg. n).minus t}, or \ff{n.neg.minus t}.

The attribute \ff{data} is used when the object is data. In this
case, the data itself is in the text of the object, as in~\lref{xml-data}.

The attribute \ff{const} denotes a constant.
